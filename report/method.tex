\section{Method}

\subsection{Discretization}
When designing the box moving agent, a variety of discretization algorithms were considered, listed as follows:

\begin{itemize}
\item Visibility graph
\item Uniform grid discretization
\item Probabilistic Roadmapping
\item Rapidly Exploring Random Trees
\end{itemize}

Uniform grid discretization was chosen as the most appropriate of these, due to the axis-aligned structure of the problem’s state.


\subsection{Search}
The search algorithm chosen for the robot box mover problem is A*

There are three aspects of a search algorithm to consider when assessing its performance in a given context - completeness, optimality, and complexity.

Judged on these three aspects, the A* algorithm implemented for this assignment has the following properties:

\begin{itemize}
\item \textbf{Completeness:} An A* algorithm is considered complete when all edges have a cost greater than ε, where ε is a small positive number. In this case, all edges do have a cost greater than 0, so the algorithm is complete, and will find a solution whenever one exists.
\item \textbf{Optimality:} An algorithm is considered optimal if it is complete, and if the heuristic chosen for the algorithm is admissible.

A heuristic, h(n), is said to be admissible if it never overestimates the cost. That is, h(n)h*(n), where h*(n) is the true cost to reach the goal from n. This is because, if a heuristic is overestimated, then that node is less likely to be expanded (due to seemingly being a potentially longer path), even if it is actually part of the shortest path to the goal. The heuristic chosen for the search at hand is admissible based on this criteria.

Since the algorithm is both complete (as stated in the previous point) and admissible, it is also considered optimal, and will return a minimum cost path whenever one exists.
\end{itemize}